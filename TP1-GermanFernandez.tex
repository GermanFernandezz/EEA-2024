% Options for packages loaded elsewhere
\PassOptionsToPackage{unicode}{hyperref}
\PassOptionsToPackage{hyphens}{url}
%
\documentclass[
]{article}
\usepackage{amsmath,amssymb}
\usepackage{iftex}
\ifPDFTeX
  \usepackage[T1]{fontenc}
  \usepackage[utf8]{inputenc}
  \usepackage{textcomp} % provide euro and other symbols
\else % if luatex or xetex
  \usepackage{unicode-math} % this also loads fontspec
  \defaultfontfeatures{Scale=MatchLowercase}
  \defaultfontfeatures[\rmfamily]{Ligatures=TeX,Scale=1}
\fi
\usepackage{lmodern}
\ifPDFTeX\else
  % xetex/luatex font selection
\fi
% Use upquote if available, for straight quotes in verbatim environments
\IfFileExists{upquote.sty}{\usepackage{upquote}}{}
\IfFileExists{microtype.sty}{% use microtype if available
  \usepackage[]{microtype}
  \UseMicrotypeSet[protrusion]{basicmath} % disable protrusion for tt fonts
}{}
\makeatletter
\@ifundefined{KOMAClassName}{% if non-KOMA class
  \IfFileExists{parskip.sty}{%
    \usepackage{parskip}
  }{% else
    \setlength{\parindent}{0pt}
    \setlength{\parskip}{6pt plus 2pt minus 1pt}}
}{% if KOMA class
  \KOMAoptions{parskip=half}}
\makeatother
\usepackage{xcolor}
\usepackage[margin=1in]{geometry}
\usepackage{color}
\usepackage{fancyvrb}
\newcommand{\VerbBar}{|}
\newcommand{\VERB}{\Verb[commandchars=\\\{\}]}
\DefineVerbatimEnvironment{Highlighting}{Verbatim}{commandchars=\\\{\}}
% Add ',fontsize=\small' for more characters per line
\usepackage{framed}
\definecolor{shadecolor}{RGB}{248,248,248}
\newenvironment{Shaded}{\begin{snugshade}}{\end{snugshade}}
\newcommand{\AlertTok}[1]{\textcolor[rgb]{0.94,0.16,0.16}{#1}}
\newcommand{\AnnotationTok}[1]{\textcolor[rgb]{0.56,0.35,0.01}{\textbf{\textit{#1}}}}
\newcommand{\AttributeTok}[1]{\textcolor[rgb]{0.13,0.29,0.53}{#1}}
\newcommand{\BaseNTok}[1]{\textcolor[rgb]{0.00,0.00,0.81}{#1}}
\newcommand{\BuiltInTok}[1]{#1}
\newcommand{\CharTok}[1]{\textcolor[rgb]{0.31,0.60,0.02}{#1}}
\newcommand{\CommentTok}[1]{\textcolor[rgb]{0.56,0.35,0.01}{\textit{#1}}}
\newcommand{\CommentVarTok}[1]{\textcolor[rgb]{0.56,0.35,0.01}{\textbf{\textit{#1}}}}
\newcommand{\ConstantTok}[1]{\textcolor[rgb]{0.56,0.35,0.01}{#1}}
\newcommand{\ControlFlowTok}[1]{\textcolor[rgb]{0.13,0.29,0.53}{\textbf{#1}}}
\newcommand{\DataTypeTok}[1]{\textcolor[rgb]{0.13,0.29,0.53}{#1}}
\newcommand{\DecValTok}[1]{\textcolor[rgb]{0.00,0.00,0.81}{#1}}
\newcommand{\DocumentationTok}[1]{\textcolor[rgb]{0.56,0.35,0.01}{\textbf{\textit{#1}}}}
\newcommand{\ErrorTok}[1]{\textcolor[rgb]{0.64,0.00,0.00}{\textbf{#1}}}
\newcommand{\ExtensionTok}[1]{#1}
\newcommand{\FloatTok}[1]{\textcolor[rgb]{0.00,0.00,0.81}{#1}}
\newcommand{\FunctionTok}[1]{\textcolor[rgb]{0.13,0.29,0.53}{\textbf{#1}}}
\newcommand{\ImportTok}[1]{#1}
\newcommand{\InformationTok}[1]{\textcolor[rgb]{0.56,0.35,0.01}{\textbf{\textit{#1}}}}
\newcommand{\KeywordTok}[1]{\textcolor[rgb]{0.13,0.29,0.53}{\textbf{#1}}}
\newcommand{\NormalTok}[1]{#1}
\newcommand{\OperatorTok}[1]{\textcolor[rgb]{0.81,0.36,0.00}{\textbf{#1}}}
\newcommand{\OtherTok}[1]{\textcolor[rgb]{0.56,0.35,0.01}{#1}}
\newcommand{\PreprocessorTok}[1]{\textcolor[rgb]{0.56,0.35,0.01}{\textit{#1}}}
\newcommand{\RegionMarkerTok}[1]{#1}
\newcommand{\SpecialCharTok}[1]{\textcolor[rgb]{0.81,0.36,0.00}{\textbf{#1}}}
\newcommand{\SpecialStringTok}[1]{\textcolor[rgb]{0.31,0.60,0.02}{#1}}
\newcommand{\StringTok}[1]{\textcolor[rgb]{0.31,0.60,0.02}{#1}}
\newcommand{\VariableTok}[1]{\textcolor[rgb]{0.00,0.00,0.00}{#1}}
\newcommand{\VerbatimStringTok}[1]{\textcolor[rgb]{0.31,0.60,0.02}{#1}}
\newcommand{\WarningTok}[1]{\textcolor[rgb]{0.56,0.35,0.01}{\textbf{\textit{#1}}}}
\usepackage{graphicx}
\makeatletter
\def\maxwidth{\ifdim\Gin@nat@width>\linewidth\linewidth\else\Gin@nat@width\fi}
\def\maxheight{\ifdim\Gin@nat@height>\textheight\textheight\else\Gin@nat@height\fi}
\makeatother
% Scale images if necessary, so that they will not overflow the page
% margins by default, and it is still possible to overwrite the defaults
% using explicit options in \includegraphics[width, height, ...]{}
\setkeys{Gin}{width=\maxwidth,height=\maxheight,keepaspectratio}
% Set default figure placement to htbp
\makeatletter
\def\fps@figure{htbp}
\makeatother
\setlength{\emergencystretch}{3em} % prevent overfull lines
\providecommand{\tightlist}{%
  \setlength{\itemsep}{0pt}\setlength{\parskip}{0pt}}
\setcounter{secnumdepth}{-\maxdimen} % remove section numbering
\ifLuaTeX
  \usepackage{selnolig}  % disable illegal ligatures
\fi
\usepackage{bookmark}
\IfFileExists{xurl.sty}{\usepackage{xurl}}{} % add URL line breaks if available
\urlstyle{same}
\hypersetup{
  pdftitle={Enfoque Estadístico del Aprendizaje},
  pdfauthor={J. Germán Fernández},
  hidelinks,
  pdfcreator={LaTeX via pandoc}}

\title{Enfoque Estadístico del Aprendizaje}
\usepackage{etoolbox}
\makeatletter
\providecommand{\subtitle}[1]{% add subtitle to \maketitle
  \apptocmd{\@title}{\par {\large #1 \par}}{}{}
}
\makeatother
\subtitle{Trabajo Práctico Nº1}
\author{J. Germán Fernández}
\date{19-10-2024}

\begin{document}
\maketitle

{
\setcounter{tocdepth}{2}
\tableofcontents
}
\subsection{Introducción}\label{introducciuxf3n}

\begin{Shaded}
\begin{Highlighting}[]
\CommentTok{\#Cargo los datos}
\NormalTok{datos }\OtherTok{\textless{}{-}} \FunctionTok{read.csv}\NormalTok{(}\StringTok{"eph\_train\_2023.csv"}\NormalTok{)}
\end{Highlighting}
\end{Shaded}

\subsection{Análisis Exploratorio de
datos}\label{anuxe1lisis-exploratorio-de-datos}

\subsubsection{Variables numericas
continuas}\label{variables-numericas-continuas}

\begin{Shaded}
\begin{Highlighting}[]
\CommentTok{\#Variables numericas{-}{-}{-}{-}}
\CommentTok{\#Estructura de los datos}
\NormalTok{datos }\SpecialCharTok{\%\textgreater{}\%} \FunctionTok{glimpse}\NormalTok{()}
\end{Highlighting}
\end{Shaded}

\begin{verbatim}
## Rows: 11,772
## Columns: 20
## $ codusu                <chr> "TQRMNOPYRHJMKPCDEIGED00785098", "TQRMNOQRXHKMLO~
## $ ano4                  <int> 2023, 2023, 2023, 2023, 2023, 2023, 2023, 2023, ~
## $ trimestre             <int> 3, 3, 3, 3, 3, 3, 3, 3, 3, 3, 3, 3, 3, 3, 3, 3, ~
## $ region                <chr> "Pampeana", "Noroeste", "Noroeste", "Gran Buenos~
## $ aglomerado            <int> 30, 25, 19, 33, 33, 93, 14, 91, 22, 36, 18, 29, ~
## $ fecha_nacimiento      <chr> "01/01/1900", "01/01/1900", "22/07/1965", "20/10~
## $ edad                  <int> 30, 28, 57, 68, 43, 39, 35, 60, 62, 32, 33, 54, ~
## $ asistencia_educacion  <chr> "asistio", "asistio", "asistio", "asistio", "asi~
## $ nivel_ed              <chr> "Secundaria\nIncompleta", "Secundaria\nCompleta"~
## $ tipo_establecimiento  <chr> "Privada", "Estatal", "Privada", "Privada", "Pri~
## $ codigo_actividad      <int> 5601, 8401, 4903, 4803, 8501, 101, 1600, 8401, 8~
## $ sexo                  <chr> "Mujer", "Varon", "Varon", "Varon", "Mujer", "Mu~
## $ categoria_ocupacion   <chr> "Obrero o empleado", "Obrero o empleado", "Obrer~
## $ cat_cantidad_empleos  <chr> "unico", "unico", "unico", "unico", "unico", "un~
## $ alfabetismo           <chr> "Sabe leer y escribir", "Sabe leer y escribir", ~
## $ salario               <int> 48000, 50000, 140000, 60000, 170000, 100000, 186~
## $ horas_trabajadas      <int> 15, 25, 63, 30, 40, 48, 40, 36, 30, 45, 20, 36, ~
## $ educacion             <int> 10, 13, 11, 8, 16, 8, 13, 8, 10, 13, 9, 15, 5, 1~
## $ experiencia_potencial <int> 15, 10, 41, 55, 22, 26, 17, 47, 47, 14, 19, 34, ~
## $ salario_horario       <dbl> 800.0000, 500.0000, 555.5556, 500.0000, 1062.500~
\end{verbatim}

\begin{Shaded}
\begin{Highlighting}[]
\NormalTok{datos }\SpecialCharTok{\%\textgreater{}\%} \FunctionTok{str}\NormalTok{()}
\end{Highlighting}
\end{Shaded}

\begin{verbatim}
## 'data.frame':    11772 obs. of  20 variables:
##  $ codusu               : chr  "TQRMNOPYRHJMKPCDEIGED00785098" "TQRMNOQRXHKMLOCDEHLEH00790365" "TQRMNOPURHLOLMCDEGPDJ00805561" "TQRMNOPTQHJMQTCDEIJAH00786499" ...
##  $ ano4                 : int  2023 2023 2023 2023 2023 2023 2023 2023 2023 2023 ...
##  $ trimestre            : int  3 3 3 3 3 3 3 3 3 3 ...
##  $ region               : chr  "Pampeana" "Noroeste" "Noroeste" "Gran Buenos Aires" ...
##  $ aglomerado           : int  30 25 19 33 33 93 14 91 22 36 ...
##  $ fecha_nacimiento     : chr  "01/01/1900" "01/01/1900" "22/07/1965" "20/10/1954" ...
##  $ edad                 : int  30 28 57 68 43 39 35 60 62 32 ...
##  $ asistencia_educacion : chr  "asistio" "asistio" "asistio" "asistio" ...
##  $ nivel_ed             : chr  "Secundaria\nIncompleta" "Secundaria\nCompleta" "Secundaria\nIncompleta" "Primaria\nCompleta" ...
##  $ tipo_establecimiento : chr  "Privada" "Estatal" "Privada" "Privada" ...
##  $ codigo_actividad     : int  5601 8401 4903 4803 8501 101 1600 8401 8401 4501 ...
##  $ sexo                 : chr  "Mujer" "Varon" "Varon" "Varon" ...
##  $ categoria_ocupacion  : chr  "Obrero o empleado" "Obrero o empleado" "Obrero o empleado" "Cuenta propia" ...
##  $ cat_cantidad_empleos : chr  "unico" "unico" "unico" "unico" ...
##  $ alfabetismo          : chr  "Sabe leer y escribir" "Sabe leer y escribir" "Sabe leer y escribir" "Sabe leer y escribir" ...
##  $ salario              : int  48000 50000 140000 60000 170000 100000 186000 150000 120000 190000 ...
##  $ horas_trabajadas     : int  15 25 63 30 40 48 40 36 30 45 ...
##  $ educacion            : int  10 13 11 8 16 8 13 8 10 13 ...
##  $ experiencia_potencial: int  15 10 41 55 22 26 17 47 47 14 ...
##  $ salario_horario      : num  800 500 556 500 1062 ...
\end{verbatim}

\begin{Shaded}
\begin{Highlighting}[]
\CommentTok{\#Quito años, trimesrte y fecha de nacimiento que no los voy a usar}
\NormalTok{datos }\SpecialCharTok{\%\textgreater{}\%} \FunctionTok{colnames}\NormalTok{()}
\end{Highlighting}
\end{Shaded}

\begin{verbatim}
##  [1] "codusu"                "ano4"                  "trimestre"            
##  [4] "region"                "aglomerado"            "fecha_nacimiento"     
##  [7] "edad"                  "asistencia_educacion"  "nivel_ed"             
## [10] "tipo_establecimiento"  "codigo_actividad"      "sexo"                 
## [13] "categoria_ocupacion"   "cat_cantidad_empleos"  "alfabetismo"          
## [16] "salario"               "horas_trabajadas"      "educacion"            
## [19] "experiencia_potencial" "salario_horario"
\end{verbatim}

\begin{Shaded}
\begin{Highlighting}[]
\NormalTok{datos }\OtherTok{\textless{}{-}}\NormalTok{ datos[}\SpecialCharTok{{-}}\FunctionTok{c}\NormalTok{(}\DecValTok{2}\NormalTok{,}\DecValTok{3}\NormalTok{,}\DecValTok{6}\NormalTok{)]}
\end{Highlighting}
\end{Shaded}


\end{document}
